\documentclass[12pt]{scrartcl}



\setlength{\parindent}{0pt}
\setlength{\parskip}{.25cm}

\usepackage{graphicx}

\usepackage{xcolor}

\definecolor{darkred}{rgb}{0.5,0,0}
\definecolor{darkgreen}{rgb}{0,0.5,0}
\usepackage{hyperref}
\hypersetup{
  letterpaper,
  colorlinks,
  linkcolor=red,
  citecolor=darkgreen,
  menucolor=darkred,
  urlcolor=blue,
  pdfpagemode=none,
  pdftitle={CS1 - Lab 6.0 - Java},
  pdfauthor={Christopher M. Bourke},
  pdfkeywords={}
}

\definecolor{MyDarkBlue}{rgb}{0,0.08,0.45}
\definecolor{MyDarkRed}{rgb}{0.45,0.08,0}
\definecolor{MyDarkGreen}{rgb}{0.08,0.45,0.08}

\definecolor{mintedBackground}{rgb}{0.95,0.95,0.95}
\definecolor{mintedInlineBackground}{rgb}{.90,.90,1}

%\usepackage{newfloat}
\usepackage[newfloat=true]{minted}
\setminted{mathescape,
               linenos,
               autogobble,
               frame=none,
               framesep=2mm,
               framerule=0.4pt,
               %label=foo,
               xleftmargin=2em,
               xrightmargin=0em,
               startinline=true,  %PHP only, allow it to omit the PHP Tags *** with this option, variables using dollar sign in comments are treated as latex math
               numbersep=10pt, %gap between line numbers and start of line
               style=default, %syntax highlighting style, default is "default"
               			    %gallery: http://help.farbox.com/pygments.html
			    	    %list available: pygmentize -L styles
               bgcolor=mintedBackground} %prevents breaking across pages
               
\setmintedinline{bgcolor={mintedBackground}}
\setminted[text]{bgcolor={mintedBackground},linenos=false,autogobble,xleftmargin=1em}
%\setminted[php]{bgcolor=mintedBackgroundPHP} %startinline=True}
\SetupFloatingEnvironment{listing}{name=Code Sample}
\SetupFloatingEnvironment{listing}{listname=List of Code Samples}

\title{CSCE 155 - Java}
\subtitle{Lab 6.0 - Methods, Enumerated\\ Types, and Exceptions}
\author{~}
\date{~}

\begin{document}

\maketitle

\section*{Prior to Lab}

Before attending this lab:
\begin{enumerate}
  \item Read and familiarize yourself with this handout.
  \item Review the following free resources:
	\begin{itemize}
	  \item Oracle's Tutorial sections on methods and returning values:\\
		\url{http://docs.oracle.com/javase/tutorial/java/javaOO/methods.html}\\
		\url{http://docs.oracle.com/javase/tutorial/java/javaOO/returnvalue.html}
	  \item Oracle's Tutorial on Exceptions:\\
		\url{http://docs.oracle.com/javase/tutorial/essential/exceptions/}
	  \item Oracle's Tutorial on enumerated types:\\
		\url{http://docs.oracle.com/javase/tutorial/java/javaOO/enum.html}
	\end{itemize}
\end{enumerate}

\section*{Peer Programming Pair-Up}

To encourage collaboration and a team environment, labs will be
structured in a \emph{pair programming} setup.  At the start of
each lab, you will be randomly paired up with another student 
(conflicts such as absences will be dealt with by the lab instructor).
One of you will be designated the \emph{driver} and the other
the \emph{navigator}.  

The navigator will be responsible for reading the instructions and
telling the driver what to do next.  The driver will be in charge of the
keyboard and workstation.  Both driver and navigator are responsible
for suggesting fixes and solutions together.  Neither the navigator
nor the driver is ``in charge.''  Beyond your immediate pairing, you
are encouraged to help and interact and with other pairs in the lab.

Each week you should alternate: if you were a driver last week, 
be a navigator next, etc.  Resolve any issues (you were both drivers
last week) within your pair.  Ask the lab instructor to resolve issues
only when you cannot come to a consensus.  

Because of the peer programming setup of labs, it is absolutely 
essential that you complete any pre-lab activities and familiarize
yourself with the handouts prior to coming to lab.  Failure to do
so will negatively impact your ability to collaborate and work with 
others which may mean that you will not be able to complete the
lab.  

\section{Lab Objectives \& Topics}
At the end of this lab you should be familiar with the following
\begin{itemize}
  \item Basics of enumerated types
  \item Writing and using methods
  \item Parameters and arguments in methods, returning values from methods
  \item How to use exceptions for error handling
  \item The proper use of enumerated types and methods in solving a given problem
\end{itemize}

\section{Background}

\subsection*{Enumerated Types}

Enumerated types are data types that define a set of named values.  Enumerated 
types are often ordered and internally associated with integers.  The purpose of 
enumerated types is to organize types that have a fixed and known set of values.  
Without enumerated types, integer values must be used and the convention of 
assigning certain integer values to certain types in a logical collection must be 
remembered or the user is consistently referring to documentation to keep 
integer values in order.  Enumerated types provide a human-readable ``tag''
to these types of elements, relieving the programmer of continually having to 
refer to the convention and avoiding errors.

In Java, enumerated types are associated with integers starting at zero.  
Enumerated types are often placed in their own source files and defined 
using the \mintinline{java}{enum} keyword (though in Java enumerated 
types are full classes).  An \mintinline{java}{enum} class itself also has several static 
convenience methods (example: \mintinline{java}{values()} returns 
a zero-index array of the enumerated types).

\subsection*{Passing Values in Functions}

In general there are two ways in which arguments can be passed to a method: 
by value (where a copy of the variable's value at the time of the method call is 
passed to the method) or by reference (a reference is passed to the method 
so that the method can manipulate its contents).  In Java however, object 
references are passed by value.

In Java, any primitive variable passed to a function is passed by value.  
However, all object variables are references.  Java passes references to 
objects slightly differently: it passes a copy of the reference to the function.  
Thus, calling an object's methods (in particular any mutators) will call the 
methods on the original object and any methods that change the state of 
the object will be realized in the calling method.  However, changing the 
actual reference only changes the local copy of the reference.  

\subsection*{Exceptions \& Error Handling}

Errors in the execution of a program are unavoidable: users may enter 
invalid input, or the expected resources (files or database connections) 
may be unavailable, etc.  Errors are communicated and handled through 
the use of exceptions.  When an error condition occurs or is detected, 
the program can throw an exception with a human-readable error message.  
The type of exception that is thrown can communicate the error in a 
programmatic manner.  When the program executes a piece of code 
that could potentially result in an exception, it can be handled by using 
a try-catch block: we try the piece of code and catch (possibly numerous 
different types of) exceptions.  We can deal with each exception in a 
different manner or simply echo (print) to the user a different message 
based on the error.  A basic example:

\begin{minted}{java}
Integer a, b;
//read in a, b
try { 
  if(b == 0) {
     throw new IllegalArgumentException("Division by " +
	"zero is not valid.\n");
  } else {
     c = a / b;
  }

  BufferedWriter out = new BufferedWriter(
	new FileWriter("/etc/passwd"));
  out.write("result = "+c);

} catch(IllegalArgumentException e1) {
     System.err.println("Division by zero is undefined!");
     System.exit(1);
} catch(SecurityException e2) {
     System.err.println("You are not root, you " +
	"can't write to the password file! ");
     System.exit(1);
}
\end{minted}

\section{Activities}

Clone the GitHub project for this lab using the following URL:
\url{https://github.com/cbourke/CSCE155-Java-Lab06}

A flower shop sells various arrangements of a dozen flowers 
(roses, lilies, daisies) in two colors each (red or white) with a 
choice of bouquet or vase.  You are given an (incomplete) source 
file, \mintinline{text}{Florist.java}, of a program that takes the order 
for a flower arrangement from the user and displays the cost. 
The program uses three enumerated types to define the type 
of flowers, color and flower arrangement.  Your task is to 
complete the program and implement the getCost function to 
compute the cost based on the following rules.
\begin{enumerate}
  \item The base price for each of the flowers is described in Table \ref{table:basePrice}
  \begin{table}[h]
  \centering
  \begin{tabular}{l|l}
  Flower & Price\\
  \hline\hline
  Roses & \$30 \\
  Lilies	 & \$20 \\
  Daisies & \$45 \\
  \end{tabular}
  \caption{Flower Prices}
  \label{table:basePrice}
  \end{table}
  \item Red Lilies and Red Daisies have an added cost of \$5 and White 
  	Roses have an added cost of \$10.  There is no additional cost for 
	other flower/color combinations
  \item Bouquets are free of charge while vases add an additional \$10 charge to the total
\end{enumerate}
	
\subsection{Defining Enumerated Types}

To complete the program, perform the following tasks.
\begin{enumerate}
  \item Insert the required items in the 3 enumerated types.  The \mintinline{java}{Color} 
  	enumerated type has already been completed for you.  
  \item After taking input from the user, the program calls the function 
	\mintinline{java}{getCost} to determine the cost. This function takes three 
	input parameters, the flower type, the color and the arrangement and 
	returns the cost (a \mintinline{java}{double}).
	\begin{enumerate}
	  \item The method declaration has been provided for you. Implement the 
	  definition of this function to return the cost of a given arrangement using 
	  the costs shown on the price list.
	  \item In the \mintinline{java}{main} method, declare the other required variable(s), complete 
	  the method call with appropriate arguments and print the cost. What's the 
	  data type of the variable cost?
	\end{enumerate}
  \item Run your program within Eclipse and answer the questions in your handout.
\end{enumerate}

\subsection{Error Handling}

Due to a harsh winter, Red Daisies are no longer available.  We will make 
the appropriate changes to the \mintinline{java}{getCost()} method to accommodate 
this change and to communicate errors by throwing an Exception.
\begin{enumerate}
  \item Check the input at the beginning of the function to check for an invalid choice.  
  	If the input is for red daisies, throw a new \mintinline{java}{IllegalArgumentException} 
	with an appropriate message.
  \item In the main method, add a try-catch block to catch the exception above and handle 
	it appropriately (simply echo an error message to the user; see the previous example).
  \item In addition, you should add another catch block to catch invalid inputs entered by 
	the user.  The type of exception thrown could be another \mintinline{java}{IllegalArgumentException} 
	with a different error message.
  \item Test your changes and demonstrate them to a lab instructor.
\end{enumerate}

%this was removed because it assumes they are using the .values() method of the enumerated type
%\subsection{Handling Different Types of Errors}
%
%Suppose the user enters an input of 7 when asked for their color choice.  What 
%would happen?  In Java a single try-catch block can have more than one catch 
%block, and can be used to check for invalid input.  Add another catch block to 
%the same try-catch block you created in Activity 2.  The type of Exception it 
%should catch is \mintinline{java}{ArrayIndexOutOfBoundsException}.  Note: 
%You do not need to  create a statement to throw the 
%\mintinline{java}{ArrayIndexOutOfBoundsException} in \mintinline{java}{getCost()}.  
%As above, simply print an error message to the user indicating that input 
%provided was invalid.
	
\section{Separating Your Code}

Up to this point, the \mintinline{java}{enum} types we used were declared in the 
class that was using them.  As mentioned earlier, enums can be their own source 
files.  Make your code more modular by organizing each of your enumerated types 
(\mintinline{java}{Flower}, \mintinline{java}{Color}, \mintinline{java}{Arrangement}) 
into their own source files.
\begin{enumerate}
  \item Create class files for each \mintinline{java}{enum}, and remove the original 
	declarations inside the \mintinline{java}{Florist} class, or comment them out
  \item Be sure to store your new \mintinline{java}{enum} classes in the same package 
	(\mintinline{java}{unl.cse.florist}) as \mintinline{text}{Florist.java}
  \item Make any necessary changes to your program and demonstrate it to a lab instructor
\end{enumerate}
	
\section{Advanced Activity (Optional)}

Large projects require even more abstraction and tools to manage source code 
and specify how it gets built.  For Java, a popular build tool is Apache Ant 
(\url{http://ant.apache.org/}).  Ant is a build utility that builds Java projects as 
specified in a special XML file (\mintinline{text}{build.xml}).  The build file specifies 
how pieces get built and the inter-dependencies on components.  Familiarize 
yourself with Ant by reading the following tutorials.
\begin{itemize}
  \item \url{http://ant.apache.org/manual/tutorial-HelloWorldWithAnt.html}
  \item \url{http://www.vogella.com/articles/ApacheAnt/article.html}
\end{itemize}
Provided in project is an example \mintinline{text}{build.xml} file.  Modify it for the 
code base you created in this lab and use it to compile and run the code from 
the command line.  In particular, from the command line:
\begin{enumerate}
  \item Place all source files into a folder named \mintinline{text}{src} in the
	same directory as the \mintinline{text}{build.xml} file (it should be like this already)
  \item Modify the \mintinline{text}{build.xml} file appropriately by specifying 
    what your main executable class is.  To do this, modify the \mintinline{text}{main.class}
    property's value to the fully qualified (full path name) class that you wish to run.
  \item Compile your project by executing the following command:
	\mintinline{text}{ant compile}
  \item Run your project by executing the following command:
	\mintinline{text}{ant run}
\end{enumerate}

\end{document}
